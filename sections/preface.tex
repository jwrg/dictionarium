\section*{Preface to the First Edition}
\addcontentsline{toc}{section}{Preface}

\begin{itemize}
  \item The modern convention of using macrons (\=a, \=o)
    instead of apices (\'a, \'o).
  \item The intention of making this document a quick reference
    for common Latin forms, and not an exhaustive compendium
  \item Also this dictionary is not meant to be a guide for the
    early beginner, but for one who already has a rudimentary
    grasp of the language.
  \item The intention is for this to be useful for me, and if others
    find it useful as well, then I'll feel like I've accomplished
    something.
  \item The core of the text comes from Wheelock and Molinarius.
    The first phase of authorship is to condense and reorganize
    the core of the course Molinari\=i to my taste.  Much of his
    work has undergone minimal paraphrasing and I am eternally in
    his debt for his expert presentation of the subject material.
    The second phase of authorship is to add everything in 
    Wheelock which is pertinent but not yet included.
  \item A third, but ongoing phase of authorship is addition;
    basically once I encounter something twice in my readings,
    it gets added until I feel like the text is a somewhat
    complete reference.
  \item Elaborating on the previous item, the first time a word
    is added to the reference, it is added but as a comment
    (in the programmatic sense, i.e., a \LaTeX~comment).  When
    it gets encountered again, the comment can simply be removed
    as I think it's likely that I'll forget how many times I come
    across a word without having some way to otherwise keep track,
    and twice seems like a reasonable threshold.
  \item Elect to eschew the lexicographer's fetish for abbreviation
    and brevity in favour of legibility and intelligibility.  It's
    not as though I am destitute on space in this document.
  \item Instead of abbreviations, leverage typography (e.g., the
    \textbackslash dictontry command which instead of displaying the terms
    side-by-side, shows them as alternating lines of raggedright
    and raggedleft, in the hopes that neither side of the entry
    takes up more than a single line.  (Did this allow for more
    columns?)
  \item Notable exception is \textit{m., f., n.,} for \textit{%
    masculine, feminine, neuter}, since I really don't want to
    type those out every time, nor do I want them to appear so
    conspicuously and ubiquitously.  Personal preference.
  \item Mulling over the idea of whether to sort verbs
    lexicographically by first-person active indicative (as
    is tradition) or to sort by infinitive, which makes
    more sense in my dumb brain for some reason (think
    \textit{d\=o, d\=are, ded\=i, datum}, all principal parts
    having different vowels after the initial \textit{d}).
  \item Also mulling over the idea of mashing the whole
    dictionary part into one section, as opposed to having
    it divided into nouns, verbs, adjectives, et cetera.
    I haven't decided which would be more useful.  If they
    are equally useful, the dictionary could conceivably
    be releases in two variations (methinks though that I
    don't much care for having more than one variation of
    the print version).
  \item I am leaning toward refining the structure to best
    exploit the attempt at separating classes of words into
    sections.  I don't know how novel this actually is but
    if I wanted a traditional dictionary, there are no
    shortages of Latin dictionaries all conforming to the
    same rules.
  \item In general, adjectives which are formed as
    participles are not called out as adjectives unless
    there is a salient reason to do so in each individual
    case.  It may also be prudent to record this reason
    at the very least as a \LaTeX~comment.
  \item It may be prudent to start thinking about what kind
    of appendices I would like to compile for the end
    matter.  Obviously a listing of irregulars would be
    nice (sum, \=idem, hoc?), and maybe some maps or a
    timeline of Roman history or literature (or both).
  \item Further appendices that may be useful include a
    colour-coded ``matrix'' of Latin colours, name
    abbreviations, counting in Latin, times and dates
    in Latin (maybe a short blurb on how our calendar
    derives essentially from that of Caesar).
\end{itemize}
