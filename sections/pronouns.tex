\section{Pronouns}

Pronouns are of particular importance in Latin, in which they are
frequently (more so than in English anyway) used to tie clauses
together in sentences and replace nouns for which the context
has been previously (explicitly) established.

\subsection{Relative Pronouns}

\begin{pronounchart}{Who, (by) whom, whose}
  \chartheading{Singular}
  Nominative  & qu\=i & quae & quod \\\hline
  Genitive    & cuius & cuius & cuius \\\hline
  Dative      & cui & cui & cui \\\hline
  Accusative  & quem & quam & quod \\\hline
  Ablative    & qu\=o & qu\=a & qu\=o \\\hline
  \chartheading{Plural}
  Nominative  & qu\=i & quae & quae \\\hline
  Genitive    & qu\=orum & qu\=arum & qu\=orum \\\hline
  Dative      & quibus & quibus & quibus \\\hline
  Accusative  & qu\=os & qu\=as & quae \\\hline
  Ablative    & quibus & quibus & quibus \\\hline
\end{pronounchart}

\subsection{Demonstrative Pronouns}
The Latin demonstrative pronouns decline much as first- and
second-declension adjectives, with the exception of the 
genitive \textit{\=ius} and dative \textit{\=i}.

\begin{pronounchart}{That, those (remote from the speaker)}
  \chartheading{Singular}
  Nominative  & ille & illa & illud \\\hline
  Genitive    & ill\=ius & ill\=ius & ill\=ius \\\hline
  Dative      & ill\=i & ill\=i & ill\=i \\\hline
  Accusative  & illum & illam & illud \\\hline
  Ablative    & ill\=o & ill\=a & ill\=o \\\hline
  \chartheading{Plural}
  Nominative  & ill\=i & illae & illa \\\hline
  Genitive    & ill\=orum & ill\=arum &\\\hline
  Dative      & ill\=is & ill\=is & ill\=is \\\hline
  Accusative  & ill\=os & ill\=as & ill\=a \\\hline
  Ablative    & ill\=is & ill\=is & ill\=is \\\hline
\end{pronounchart}

\begin{pronounchart}{this, that, he, she, it, these, those, they (unemphatic)}
  \chartheading{Singular}
  Nominative  & is & ea & id \\\hline
  Genitive    & eius & eius & eius \\\hline
  Dative      & e\=i & e\=i & e\=i \\\hline
  Accusative  & eum & eam & id \\\hline
  Ablative    & e\=o & e\=a & e\=o \\\hline
  \chartheading{Plural}
  Nominative  & e\=i & eae & ea \\\hline
  Genitive    & e\=orum & e\=arum & e\=orum \\\hline
  Dative      & e\=is & e\=is & e\=is \\\hline
  Accusative  & e\=os & e\=as & ea \\\hline
  Ablative    & e\=is & e\=is & e\=is \\\hline
\end{pronounchart}

Particularly irregular is \textbf{hic, haec, hoc}, which,
especially in the singular, marches to its own tune.
\begin{pronounchart}{This, these (proximal to the speaker)}
  \chartheading{Singular}
  Nominative  & hic & haec & hoc \\\hline
  Genitive    & huius & huius & huius \\\hline
  Dative      & huic & huic & huic \\\hline
  Accusative  & hunc & hanc & hoc \\\hline
  Ablative    & h\=oc & h\=ac & h\=oc \\\hline
  \chartheading{Plural}
  Nominative  & h\=i & hae & haec \\\hline
  Genitive    & h\=orum & h\=arum & h\=orum \\\hline
  Dative      & h\=is & h\=is & h\=is \\\hline
  Accusative  & h\=os & h\=as & haec \\\hline
  Ablative    & h\=is & h\=is & h\=is \\\hline
\end{pronounchart}

\subsection{Demonstrative Adjectives}
A class of adjectives which can be used in place of nouns,
i.e., demonstrative, decline much as the demonstrative
pronouns above; these are identified in dictionary entries
with a note in parentheses.  Examples of such are:
\begin{multicols}{3}
  \setlength{\columnseprule}{0pt}
  \textbf{
  alter, altera, alterum \\
  alius, alia, aliud \\
  ipse, ipsa, ipsum \\
  iste, ista, istud \\ \vfill\null\columnbreak
  idem, eadem, idem \\
  s\=olus, s\=ola, s\=olum \\
  t\=otus, t\=ota, t\=otum \\
  \=ullus, \=ulla, \=ullum \\ \vfill\null\columnbreak
  n\=ullus, n\=ulla, n\=ullum \\
  \=unus, \=una, \=unum \\
  uter, utra, utrum \\
  neuter, neutra, neutrum \\ \vfill\null\columnbreak
  }
\end{multicols}
\subsection{Testes}
