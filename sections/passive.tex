\section{The passive voice}
The degree to which the forms of the active and passive voices
in Latin differ warrant that a full section should be devoted 
to conjugate each voice.  This section comes after the
description of participles owing to the ubiquitous use of 
participles by the passive voice. This section will describe 
verb forms constructed using the passive voice.

\subsection{Present-tense paradigm forms}
The present-tense system, based on the first principal part, is mostly
irregular, with degrees of variance across the conjugations.  The only
endings common to all conjugations are given here, namely the
\textit{present passive indicative}.

\begin{verbchart}{Personal endings}{present}{passive}{indicative}
  1 & -\textbf{r}   & -\textbf{mur}     \\\hline
  2 & -\textbf{ris} & -\textbf{min\=i}  \\\hline
  3 & -\textbf{tur} & -\textbf{ntur}    \\\hline
\end{verbchart}

% Need to describe the imperfect and future here as well.  It's
% much the same as for the present-tense forms, so do that one
% first.

\subsection{Perfect-tense paradigm forms}
The perfect-tense system in particular differs much from other
constructions in that all the perfect-tense passive-voice forms
have two parts, namely the nominative past participle, followed by
a form of \emph{esse}, with the tense of the final construction
being indicated by the tense of the form of \emph{esse} used in
the construction.

NB that a full conjugation for \emph{esse} and other irregular 
verbs can be found in Appendix XXX.

\begin{verbchart}{Personal endings}{perfect}{passive}{indicative}
  1 & -\textbf{us}, -\textbf{a}, -\textbf{um sum}
    & -\textbf{i}, -\textbf{ae}, -\textbf{a sumus} \\\hline
  2 & -\textbf{us}, -\textbf{a}, -\textbf{um es}
    & -\textbf{i}, -\textbf{ae}, -\textbf{a estis} \\\hline
  3 & -\textbf{us}, -\textbf{a}, -\textbf{um est}
    & -\textbf{i}, -\textbf{ae}, -\textbf{a sunt} \\\hline
\end{verbchart}

\begin{verbchart}{Personal endings}{perfect}{passive}{subjuntive}
  1 & -\textbf{us}, -\textbf{a}, -\textbf{um sim}
    & -\textbf{i}, -\textbf{ae}, -\textbf{a s\=imus} \\\hline
  2 & -\textbf{us}, -\textbf{a}, -\textbf{um s\=is}
    & -\textbf{i}, -\textbf{ae}, -\textbf{a s\=itis} \\\hline
  3 & -\textbf{us}, -\textbf{a}, -\textbf{um sit}
    & -\textbf{i}, -\textbf{ae}, -\textbf{a sint} \\\hline
\end{verbchart}

\begin{verbchart}{Personal endings}{pluperfect}{passive}{indicative}
  1 & -\textbf{us}, -\textbf{a}, -\textbf{um eram}
    & -\textbf{i}, -\textbf{ae}, -\textbf{a er\=amus} \\\hline
  2 & -\textbf{us}, -\textbf{a}, -\textbf{um er\=as}
    & -\textbf{i}, -\textbf{ae}, -\textbf{a er\=atis} \\\hline
  3 & -\textbf{us}, -\textbf{a}, -\textbf{um erat}
    & -\textbf{i}, -\textbf{ae}, -\textbf{a erant} \\\hline
\end{verbchart}

\begin{verbchart}{Personal endings}{pluperfect}{passive}{subjunctive}
  1 & -\textbf{us}, -\textbf{a}, -\textbf{um essem}
    & -\textbf{i}, -\textbf{ae}, -\textbf{a ess\=emus} \\\hline
  2 & -\textbf{us}, -\textbf{a}, -\textbf{um ess\=es}
    & -\textbf{i}, -\textbf{ae}, -\textbf{a ess\=etis} \\\hline
  3 & -\textbf{us}, -\textbf{a}, -\textbf{um esset}
    & -\textbf{i}, -\textbf{ae}, -\textbf{a essent} \\\hline
\end{verbchart}

\begin{verbchart}{Personal endings}{future perfect}{passive}{indicative}
  1 & -\textbf{us}, -\textbf{a}, -\textbf{um er\=o}
    & -\textbf{i}, -\textbf{ae}, -\textbf{a erimus} \\\hline
  2 & -\textbf{us}, -\textbf{a}, -\textbf{um eris}
    & -\textbf{i}, -\textbf{ae}, -\textbf{a eritis} \\\hline
  3 & -\textbf{us}, -\textbf{a}, -\textbf{um erit}
    & -\textbf{i}, -\textbf{ae}, -\textbf{a erunt} \\\hline
\end{verbchart}

\subsection{The First Conjugation (Long-\=a verbs)}
First conjugation verbs are characterized by the theme vowel, \=a.  Unlike
all other conjugations, the subjunctive is formed with a long \=e.

\paragraph{Present-tense paradigm forms}

\begin{verbchart}{coni\=ur\=o, coni\=ur\=are, coni\=ur\=avi, coni\=ur\=atum}{present}{passive}{indicative}
  1 & coni\=uro\textbf{r}     & coni\=ur\=a\textbf{mur} \\\hline
  2 & coni\=ur\=a\textbf{ris} & coni\=ur\=a\textbf{min\=i} \\\hline
  3 & coni\=ura\textbf{tur}   & coni\=ura\textbf{ntur} \\\hline
\end{verbchart}

\begin{verbchart}{laud\=o, laud\=are, laud\=avi, laud\=atum}{present}{passive}{subjunctive}
  1 & laud\textbf{er}     & laud\textbf{\=emur} \\\hline
  2 & laud\textbf{\=eris} & laud\textbf{\=emin\=i} \\\hline
  3 & laud\textbf{\=etur} & laud\textbf{entur} \\\hline
\end{verbchart}

\begin{verbchart}{coni\=ur\=o, coni\=ur\=are, coni\=ur\=avi, coni\=ur\=atum}{future}{passive}{indicative}
  1 & coni\=ur\=a\textbf{bor}     & coni\=ur\=a\textbf{bimur} \\\hline
  2 & coni\=ur\=a\textbf{beris}   & coni\=ur\=a\textbf{bimin\=i} \\\hline
  3 & coni\=ur\=a\textbf{bitur}   & coni\=ur\=a\textbf{buntur} \\\hline
\end{verbchart}

\begin{verbchart}{coni\=ur\=o, coni\=ur\=are, coni\=ur\=avi, coni\=ur\=atum}{imperfect}{passive}{indicative}
  1 & coni\=ur\=a\textbf{bar}     & coni\=ur\=a\textbf{b\=amur} \\\hline
  2 & coni\=ur\=a\textbf{b\=aris} & coni\=ur\=a\textbf{b\=amin\=i} \\\hline
  3 & coni\=ur\=a\textbf{b\=atur} & coni\=ur\=a\textbf{b\=antur} \\\hline
\end{verbchart}

\paragraph{Perfect-tense paradigm forms}

\begin{verbchart}{Personal endings}{perfect}{passive}{indicative}
  1 & am\=at\textbf{us}, am\=at\textbf{a},  am\=at\textbf{um sum}
    & am\=at\textbf{i},  am\=at\textbf{ae}, am\=at\textbf{a sumus} \\\padline
  2 & am\=at\textbf{us}, am\=at\textbf{a},  am\=at\textbf{um es}
    & am\=at\textbf{i},  am\=at\textbf{ae}, am\=at\textbf{a estis} \\\padline
  3 & am\=at\textbf{us}, am\=at\textbf{a},  am\=at\textbf{um est}
    & am\=at\textbf{i},  am\=at\textbf{ae}, am\=at\textbf{a sunt} \par \\\hline
\end{verbchart}

\begin{verbchart}{Personal endings}{perfect}{passive}{subjuntive}
  1 & am\=at\textbf{us},  am\=at\textbf{a},  am\=at\textbf{um sim}
    & am\=at\textbf{i},   am\=at\textbf{ae}, am\=at\textbf{a s\=imus} \\\padline
  2 & am\=at\textbf{us},  am\=at\textbf{a},  am\=at\textbf{um s\=is}
    & am\=at\textbf{i},   am\=at\textbf{ae}, am\=at\textbf{a s\=itis} \\\padline
  3 & am\=at\textbf{us},  am\=at\textbf{a},  am\=at\textbf{um sit}
    & am\=at\textbf{i},   am\=at\textbf{ae}, am\=at\textbf{a sint} \par \\\hline
\end{verbchart}

\begin{verbchart}{Personal endings}{pluperfect}{passive}{indicative}
  1 & am\=at\textbf{us},  am\=at\textbf{a},  am\=at\textbf{um eram}
    & am\=at\textbf{i},   am\=at\textbf{ae}, am\=at\textbf{a er\=amus} \\\padline
  2 & am\=at\textbf{us},  am\=at\textbf{a},  am\=at\textbf{um er\=as}
    & am\=at\textbf{i},   am\=at\textbf{ae}, am\=at\textbf{a er\=atis} \\\padline
  3 & am\=at\textbf{us},  am\=at\textbf{a},  am\=at\textbf{um erat}
    & am\=at\textbf{i},   am\=at\textbf{ae}, am\=at\textbf{a erant} \par \\\hline
\end{verbchart}

\begin{verbchart}{Personal endings}{pluperfect}{passive}{subjunctive}
  1 & am\=at\textbf{us},  am\=at\textbf{a},  am\=at\textbf{um essem}
    & am\=at\textbf{i},   am\=at\textbf{ae}, am\=at\textbf{a ess\=emus} \\\padline
  2 & am\=at\textbf{us},  am\=at\textbf{a},  am\=at\textbf{um ess\=es}
    & am\=at\textbf{i},   am\=at\textbf{ae}, am\=at\textbf{a ess\=etis} \\\padline
  3 & am\=at\textbf{us},  am\=at\textbf{a},  am\=at\textbf{um esset}
    & am\=at\textbf{i},   am\=at\textbf{ae}, am\=at\textbf{a essent} \par \\\hline
\end{verbchart}

\begin{verbchart}{Personal endings}{future perfect}{passive}{indicative}
  1 & am\=at\textbf{us},  am\=at\textbf{a},  am\=at\textbf{um er\=o}
    & am\=at\textbf{i},   am\=at\textbf{ae}, am\=at\textbf{a erimus} \\\padline
  2 & am\=at\textbf{us},  am\=at\textbf{a},  am\=at\textbf{um eris}
    & am\=at\textbf{i},   am\=at\textbf{ae}, am\=at\textbf{a eritis} \\\padline
  3 & am\=at\textbf{us},  am\=at\textbf{a},  am\=at\textbf{um erit}
    & am\=at\textbf{i},   am\=at\textbf{ae}, am\=at\textbf{a erunt} \par \\\hline
\end{verbchart}

\subsection{The Second Conjugation (Long-\=e verbs)}
Second conjugations verbs are characterized by the theme vowel, \=e.  Easily
recognizable second conjugation verbs include \textit{vide\=o, vid\=ere,
v\=id\=i, v\=isum}, and \textit{habe\=o, hab\=ere, habu\=i, habitum}.

\paragraph{Present-tense paradigm forms}

\begin{verbchart}{fove\=o, fov\=ere, fov\=i, f\=otum}{present}{passive}{indicative}
  1 & foveo\textbf{r}    & fov\=e\textbf{mur} \\\hline
  2 & fov\=e\textbf{ris} & fov\=e\textbf{min\=i} \\\hline
  3 & fove\textbf{tur}   & fove\textbf{ntur} \\\hline
\end{verbchart}

Unlike the first and third conjugations, and like the fourth conjugation,
the subjunctive for second conjugation verbs is formed with a long \=a 
\textit{after} a shortened theme vowel, in this case short-e.

\begin{verbchart}{fove\=o, fov\=ere, fov\=i, f\=otum}{present}{passive}{subjunctive}
  1 & fove\textbf{ar}      & fove\textbf{\=amur} \\\hline
  2 & fove\textbf{\=aris}  & fove\textbf{\=amin\=i} \\\hline
  3 & fove\textbf{\=atur}  & fove\textbf{antur} \\\hline
\end{verbchart}

\begin{verbchart}{fove\=o, fov\=ere, fov\=i, f\=otum}{future}{passive}{indicative}
  1 & fov\=e\textbf{bor}    & fov\=e\textbf{bimur} \\\hline
  2 & fov\=e\textbf{beris}  & fov\=e\textbf{bimin\=i} \\\hline
  3 & fov\=e\textbf{bitur}  & fov\=e\textbf{buntur} \\\hline
\end{verbchart}

\begin{verbchart}{fove\=o, fov\=ere, fov\=i, f\=otum}{imperfect}{passive}{indicative}
  1 & fov\=e\textbf{bar}      & fov\=e\textbf{b\=amur} \\\hline
  2 & fov\=e\textbf{b\=aris}  & fov\=e\textbf{b\=amin\=i} \\\hline
  3 & fov\=e\textbf{b\=atur}  & fov\=e\textbf{bantur} \\\hline
\end{verbchart}

\paragraph{Perfect-tense paradigm forms}

\begin{verbchart}{Personal endings}{perfect}{passive}{indicative}
  1 & v\=is\textbf{us}, v\=is\textbf{a}, v\=is\textbf{um sum}
    & v\=is\textbf{i}, v\=is\textbf{ae}, v\=is\textbf{a sumus} \\\padline
  2 & v\=is\textbf{us}, v\=is\textbf{a}, v\=is\textbf{um es}
    & v\=is\textbf{i}, v\=is\textbf{ae}, v\=is\textbf{a estis} \\\padline
  3 & v\=is\textbf{us}, v\=is\textbf{a}, v\=is\textbf{um est}
    & v\=is\textbf{i}, v\=is\textbf{ae}, v\=is\textbf{a sunt} \par \\\hline
\end{verbchart}

\begin{verbchart}{Personal endings}{perfect}{passive}{subjunctive}
  1 & v\=is\textbf{us}, v\=is\textbf{a}, v\=is\textbf{um sim}
    & v\=is\textbf{i}, v\=is\textbf{ae}, v\=is\textbf{a s\=imus} \\\padline
  2 & v\=is\textbf{us}, v\=is\textbf{a}, v\=is\textbf{um s\=is}
    & v\=is\textbf{i}, v\=is\textbf{ae}, v\=is\textbf{a s\=itis} \\\padline
  3 & v\=is\textbf{us}, v\=is\textbf{a}, v\=is\textbf{um sit}
    & v\=is\textbf{i}, v\=is\textbf{ae}, v\=is\textbf{a sint} \par \\\hline
\end{verbchart}

\begin{verbchart}{Personal endings}{pluperfect}{passive}{indicative}
  1 & v\=is\textbf{us}, v\=is\textbf{a}, v\=is\textbf{um eram}
    & v\=is\textbf{i}, v\=is\textbf{ae}, v\=is\textbf{a er\=amus} \\\padline
  2 & v\=is\textbf{us}, v\=is\textbf{a}, v\=is\textbf{um er\=as}
    & v\=is\textbf{i}, v\=is\textbf{ae}, v\=is\textbf{a er\=atis} \\\padline
  3 & v\=is\textbf{us}, v\=is\textbf{a}, v\=is\textbf{um erat}
    & v\=is\textbf{i}, v\=is\textbf{ae}, v\=is\textbf{a erant} \par \\\hline
\end{verbchart}

\begin{verbchart}{Personal endings}{pluperfect}{passive}{subjunctive}
  1 & v\=is\textbf{us}, v\=is\textbf{a}, v\=is\textbf{um essem}
    & v\=is\textbf{i}, v\=is\textbf{ae}, v\=is\textbf{a ess\=emus} \\\padline
  2 & v\=is\textbf{us}, v\=is\textbf{a}, v\=is\textbf{um ess\=es}
    & v\=is\textbf{i}, v\=is\textbf{ae}, v\=is\textbf{a ess\=etis} \\\padline
  3 & v\=is\textbf{us}, v\=is\textbf{a}, v\=is\textbf{um esset}
    & v\=is\textbf{i}, v\=is\textbf{ae}, v\=is\textbf{a essent} \par \\\hline
\end{verbchart}

\begin{verbchart}{Personal endings}{future perfect}{passive}{indicative}
  1 & v\=is\textbf{us}, v\=is\textbf{a}, v\=is\textbf{um er\=o}
    & v\=is\textbf{i}, v\=is\textbf{ae}, v\=is\textbf{a erimus} \\\padline
  2 & v\=is\textbf{us}, v\=is\textbf{a}, v\=is\textbf{um eris}
    & v\=is\textbf{i}, v\=is\textbf{ae}, v\=is\textbf{a eritis} \\\padline
  3 & v\=is\textbf{us}, v\=is\textbf{a}, v\=is\textbf{um erit}
    & v\=is\textbf{i}, v\=is\textbf{ae}, v\=is\textbf{a erunt} \par \\\hline
\end{verbchart}

\subsection{The Third Conjugation (Short-e verbs)}
The third conjugation of Latin verbs is arguably the least regular, and is
characterized by the theme vowel short-e.  Notably, this short-e tends to
manifest itself as a short-i, as in the present active indicative.

\paragraph{Present-tense paradigm forms}

\begin{verbchart}{ag\=o, agere, \=eg\=i, \=actum}{present}{passive}{indicative}
  1 & ago\textbf{r}     & agi\textbf{mur} \\\hline
  2 & age\textbf{ris}   & agi\textbf{min\=i} \\\hline
  3 & agi\textbf{tur}   & agu\textbf{ntur} \\\hline
\end{verbchart}

\begin{verbchart}{ed\=o, edere, \=ed\=i, esum}{present}{passive}{subjunctive}
  1 & ed\textbf{ar}     & ed\textbf{\=amur} \\\hline
  2 & ed\textbf{\=aris} & ed\=a\textbf{min\=i} \\\hline
  3 & ed\textbf{\=atur} & ed\textbf{antur} \\\hline
\end{verbchart}

\begin{verbchart}{ed\=o, edere, \=ed\=i, esum}{future}{passive}{indicative}
  1 & ed\textbf{ar}     & ed\textbf{\=emur} \\\hline
  2 & ed\textbf{\=eris} & ed\=a\textbf{\=emin\=i} \\\hline
  3 & ed\textbf{\=etur} & ed\textbf{entur} \\\hline
\end{verbchart}

\begin{verbchart}{ag\=o, agere, \=eg\=i, \=actum}{imperfect}{passive}{indicative}
  1 & ag\=e\textbf{bar}     & ag\=e\textbf{b\=amur} \\\hline
  2 & ag\=e\textbf{b\=aris} & ag\=e\textbf{b\=amin\=i} \\\hline
  3 & ag\=e\textbf{b\=atur} & ag\=e\textbf{bantur} \\\hline
\end{verbchart}

\paragraph{Perfect-tense paradigm forms}

\begin{verbchart}{Personal endings}{perfect}{passive}{indicative}
  1 & posit\textbf{us}, posit\textbf{a}, posit\textbf{um sum}
    & posit\textbf{i}, posit\textbf{ae}, posit\textbf{a sumus} \\\padline
  2 & posit\textbf{us}, posit\textbf{a}, posit\textbf{um es}
    & posit\textbf{i}, posit\textbf{ae}, posit\textbf{a estis} \\\padline
  3 & posit\textbf{us}, posit\textbf{a}, posit\textbf{um est}
    & posit\textbf{i}, posit\textbf{ae}, posit\textbf{a sunt} \par \\\hline
\end{verbchart}

\begin{verbchart}{Personal endings}{perfect}{passive}{subjuntive}
  1 & posit\textbf{us}, posit\textbf{a}, posit\textbf{um sim}
    & posit\textbf{i}, posit\textbf{ae}, posit\textbf{a s\=imus} \\\padline
  2 & posit\textbf{us}, posit\textbf{a}, posit\textbf{um s\=is}
    & posit\textbf{i}, posit\textbf{ae}, posit\textbf{a s\=itis} \\\padline
  3 & posit\textbf{us}, posit\textbf{a}, posit\textbf{um sit}
    & posit\textbf{i}, posit\textbf{ae}, posit\textbf{a sint} \par \\\hline
\end{verbchart}

\begin{verbchart}{Personal endings}{pluperfect}{passive}{indicative}
  1 & posit\textbf{us}, posit\textbf{a}, posit\textbf{um eram}
    & posit\textbf{i}, posit\textbf{ae}, posit\textbf{a er\=amus} \\\padline
  2 & posit\textbf{us}, posit\textbf{a}, posit\textbf{um er\=as}
    & posit\textbf{i}, posit\textbf{ae}, posit\textbf{a er\=atis} \\\padline
  3 & posit\textbf{us}, posit\textbf{a}, posit\textbf{um erat}
    & posit\textbf{i}, posit\textbf{ae}, posit\textbf{a erant} \par \\\hline
\end{verbchart}

\begin{verbchart}{Personal endings}{pluperfect}{passive}{subjunctive}
  1 & posit\textbf{us}, posit\textbf{a}, posit\textbf{um essem}
    & posit\textbf{i}, posit\textbf{ae}, posit\textbf{a ess\=emus} \\\padline
  2 & posit\textbf{us}, posit\textbf{a}, posit\textbf{um ess\=es}
    & posit\textbf{i}, posit\textbf{ae}, posit\textbf{a ess\=etis} \\\padline
  3 & posit\textbf{us}, posit\textbf{a}, posit\textbf{um esset}
    & posit\textbf{i}, posit\textbf{ae}, posit\textbf{a essent} \par \\\hline
\end{verbchart}

\begin{verbchart}{Personal endings}{future perfect}{passive}{indicative}
  1 & posit\textbf{us}, posit\textbf{a}, posit\textbf{um er\=o}
    & posit\textbf{i}, posit\textbf{ae}, posit\textbf{a erimus} \\\padline
  2 & posit\textbf{us}, posit\textbf{a}, posit\textbf{um eris}
    & posit\textbf{i}, posit\textbf{ae}, posit\textbf{a eritis} \\\padline
  3 & posit\textbf{us}, posit\textbf{a}, posit\textbf{um erit}
    & posit\textbf{i}, posit\textbf{ae}, posit\textbf{a erunt} \par \\\hline
\end{verbchart}

\subsection{The Third\textit{-io}}
A subset of the third conjugation, the third-io can
be thought of as part third- and part fourth-conjugation.
It retains the theme vowel short-e but places an i in
some forms where it is not expected for a third conjugation
verb.

\paragraph{Present-tense paradigm forms}

\begin{verbchart}{capi\=o, capere, c\=ep\=i, captum}{present}{passive}{indicative}
  1 & capio\textbf{r}   & capi\textbf{mur} \\\hline
  2 & capie\textbf{ris} & capi\textbf{min\=i} \\\hline
  3 & capi\textbf{tur}  & capiu\textbf{ntur} \\\hline
\end{verbchart}

\begin{verbchart}{capi\=o, capere, c\=ep\=i, captum}{present}{passive}{subjunctive}
  1 & capi\textbf{ar}     & capi\textbf{\=amur} \\\hline
  2 & capi\textbf{\=aris} & capi\textbf{\=amin\=i} \\\hline
  3 & capi\textbf{\=atur} & capi\textbf{antur} \\\hline
\end{verbchart}

\begin{verbchart}{capi\=o, capere, c\=ep\=i, captum}{future}{passive}{indicative}
  1 & capi\textbf{ar}     & capi\textbf{\=emur} \\\hline
  2 & capi\textbf{\=eris} & capi\textbf{\=emin\=i} \\\hline
  3 & capi\textbf{\=etur} & capi\textbf{entur} \\\hline
\end{verbchart}

\begin{verbchart}{capi\=o, capere, c\=ep\=i, captum}{imperfect}{passive}{indicative}
  1 & capi\=e\textbf{bar}     & capi\=e\textbf{b\=amur} \\\hline
  2 & capi\=e\textbf{b\=aris} & capi\=e\textbf{b\=amin\=i} \\\hline
  3 & capi\=e\textbf{b\=atur} & capi\=e\textbf{bantur} \\\hline
\end{verbchart}

\paragraph{Perfect-tense paradigm forms}

\begin{verbchart}{Personal endings}{perfect}{passive}{indicative}
  1 & capt\textbf{us}, capt\textbf{a}, capt\textbf{um sum}
    & capt\textbf{i}, capt\textbf{ae}, capt\textbf{a sumus} \\\padline
  2 & capt\textbf{us}, capt\textbf{a}, capt\textbf{um es}
    & capt\textbf{i}, capt\textbf{ae}, capt\textbf{a estis} \\\padline
  3 & capt\textbf{us}, capt\textbf{a}, capt\textbf{um est}
    & capt\textbf{i}, capt\textbf{ae}, capt\textbf{a sunt} \par \\\hline
\end{verbchart}

\begin{verbchart}{Personal endings}{perfect}{passive}{subjuntive}
  1 & capt\textbf{us}, capt\textbf{a}, capt\textbf{um sim}
    & capt\textbf{i}, capt\textbf{ae}, capt\textbf{a s\=imus} \\\padline
  2 & capt\textbf{us}, capt\textbf{a}, capt\textbf{um s\=is}
    & capt\textbf{i}, capt\textbf{ae}, capt\textbf{a s\=itis} \\\padline
  3 & capt\textbf{us}, capt\textbf{a}, capt\textbf{um sit}
    & capt\textbf{i}, capt\textbf{ae}, capt\textbf{a sint} \par \\\hline
\end{verbchart}

\begin{verbchart}{Personal endings}{pluperfect}{passive}{indicative}
  1 & capt\textbf{us}, capt\textbf{a}, capt\textbf{um eram}
    & capt\textbf{i}, capt\textbf{ae}, capt\textbf{a er\=amus} \\\padline
  2 & capt\textbf{us}, capt\textbf{a}, capt\textbf{um er\=as}
    & capt\textbf{i}, capt\textbf{ae}, capt\textbf{a er\=atis} \\\padline
  3 & capt\textbf{us}, capt\textbf{a}, capt\textbf{um erat}
    & capt\textbf{i}, capt\textbf{ae}, capt\textbf{a erant} \par \\\hline
\end{verbchart}

\begin{verbchart}{Personal endings}{pluperfect}{passive}{subjunctive}
  1 & capt\textbf{us}, capt\textbf{a}, capt\textbf{um essem}
    & capt\textbf{i}, capt\textbf{ae}, capt\textbf{a ess\=emus} \\\padline
  2 & capt\textbf{us}, capt\textbf{a}, capt\textbf{um ess\=es}
    & capt\textbf{i}, capt\textbf{ae}, capt\textbf{a ess\=etis} \\\padline
  3 & capt\textbf{us}, capt\textbf{a}, capt\textbf{um esset}
    & capt\textbf{i}, capt\textbf{ae}, capt\textbf{a essent} \par \\\hline
\end{verbchart}

\begin{verbchart}{Personal endings}{future perfect}{passive}{indicative}
  1 & capt\textbf{us}, capt\textbf{a}, capt\textbf{um er\=o}
    & capt\textbf{i}, capt\textbf{ae}, capt\textbf{a erimus} \\\padline
  2 & capt\textbf{us}, capt\textbf{a}, capt\textbf{um eris}
    & capt\textbf{i}, capt\textbf{ae}, capt\textbf{a eritis} \\\padline
  3 & capt\textbf{us}, capt\textbf{a}, capt\textbf{um erit}
    & capt\textbf{i}, capt\textbf{ae}, capt\textbf{a erunt} \par \\\hline
\end{verbchart}

\subsection{The Fourth Conjugation (Long-\=i verbs)}
The fourth conjugation is characterized by the theme vowel
long-\=i.

\paragraph{Present-tense paradigm forms}

\begin{verbchart}{senti\=o, sent\=ire, s\=ens\=i, s\=ensum}{present}{passive}{indicative}
  1 & sentio\textbf{r}      & sent\=i\textbf{mur} \\\hline
  2 & sent\=i\textbf{ris}   & sent\=i\textbf{min\=i} \\\hline
  3 & sent\=i\textbf{tur}   & sentiu\textbf{ntur} \\\hline
\end{verbchart}

\begin{verbchart}{senti\=o, sent\=ire, s\=ens\=i, s\=ensum}{present}{passive}{subjunctive}
  1 & senti\textbf{ar}       & senti\textbf{\=amur} \\\hline
  2 & senti\textbf{\=aris}   & senti\textbf{\=amin\=i} \\\hline
  3 & senti\textbf{\=atur}   & senti\textbf{antur} \\\hline
\end{verbchart}

\begin{verbchart}{senti\=o, sent\=ire, s\=ens\=i, s\=ensum}{future}{passive}{indicative}
  1 & senti\textbf{ar}      & senti\textbf{\=emur} \\\hline
  2 & senti\textbf{\=eris}  & senti\textbf{\=emin\=i} \\\hline
  3 & senti\textbf{\=etur}  & senti\textbf{entur} \\\hline
\end{verbchart}

\begin{verbchart}{senti\=o, sent\=ire, s\=ens\=i, s\=ensum}{imperfect}{passive}{indicative}
  1 & senti\=e\textbf{bar}      & senti\=e\textbf{b\=amur} \\\hline
  2 & senti\=e\textbf{b\=aris}  & senti\=e\textbf{b\=amin\=i} \\\hline
  3 & sentI\=e\textbf{b\=atur}  & senti\=e\textbf{bantur} \\\hline
\end{verbchart}

\paragraph{Perfect-tense paradigm forms}

\begin{verbchart}{Personal endings}{perfect}{passive}{indicative}
  1 & s\=ens\textbf{us}, s\=ens\textbf{a}, s\=ens\textbf{um sum}
    & s\=ens\textbf{i}, s\=ens\textbf{ae}, s\=ens\textbf{a sumus} \\\padline
  2 & s\=ens\textbf{us}, s\=ens\textbf{a}, s\=ens\textbf{um es}
    & s\=ens\textbf{i}, s\=ens\textbf{ae}, s\=ens\textbf{a estis} \\\padline
  3 & s\=ens\textbf{us}, s\=ens\textbf{a}, s\=ens\textbf{um est}
    & s\=ens\textbf{i}, s\=ens\textbf{ae}, s\=ens\textbf{a sunt} \par \\\hline
\end{verbchart}

\begin{verbchart}{Personal endings}{perfect}{passive}{subjuntive}
  1 & s\=ens\textbf{us}, s\=ens\textbf{a}, s\=ens\textbf{um sim}
    & s\=ens\textbf{i}, s\=ens\textbf{ae}, s\=ens\textbf{a s\=imus} \\\padline
  2 & s\=ens\textbf{us}, s\=ens\textbf{a}, s\=ens\textbf{um s\=is}
    & s\=ens\textbf{i}, s\=ens\textbf{ae}, s\=ens\textbf{a s\=itis} \\\padline
  3 & s\=ens\textbf{us}, s\=ens\textbf{a}, s\=ens\textbf{um sit}
    & s\=ens\textbf{i}, s\=ens\textbf{ae}, s\=ens\textbf{a sint} \par \\\hline
\end{verbchart}

\begin{verbchart}{Personal endings}{pluperfect}{passive}{indicative}
  1 & s\=ens\textbf{us}, s\=ens\textbf{a}, s\=ens\textbf{um eram}
    & s\=ens\textbf{i}, s\=ens\textbf{ae}, s\=ens\textbf{a er\=amus} \\\padline
  2 & s\=ens\textbf{us}, s\=ens\textbf{a}, s\=ens\textbf{um er\=as}
    & s\=ens\textbf{i}, s\=ens\textbf{ae}, s\=ens\textbf{a er\=atis} \\\padline
  3 & s\=ens\textbf{us}, s\=ens\textbf{a}, s\=ens\textbf{um erat}
    & s\=ens\textbf{i}, s\=ens\textbf{ae}, s\=ens\textbf{a erant} \par \\\hline
\end{verbchart}

\begin{verbchart}{Personal endings}{pluperfect}{passive}{subjunctive}
  1 & s\=ens\textbf{us}, s\=ens\textbf{a}, s\=ens\textbf{um essem}
    & s\=ens\textbf{i}, s\=ens\textbf{ae}, s\=ens\textbf{a ess\=emus} \\\padline
  2 & s\=ens\textbf{us}, s\=ens\textbf{a}, s\=ens\textbf{um ess\=es}
    & s\=ens\textbf{i}, s\=ens\textbf{ae}, s\=ens\textbf{a ess\=etis} \\\padline
  3 & s\=ens\textbf{us}, s\=ens\textbf{a}, s\=ens\textbf{um esset}
    & s\=ens\textbf{i}, s\=ens\textbf{ae}, s\=ens\textbf{a essent} \par \\\hline
\end{verbchart}

\begin{verbchart}{Personal endings}{future perfect}{passive}{indicative}
  1 & s\=ens\textbf{us}, s\=ens\textbf{a}, s\=ens\textbf{um er\=o}
    & s\=ens\textbf{i}, s\=ens\textbf{ae}, s\=ens\textbf{a erimus} \\\padline
  2 & s\=ens\textbf{us}, s\=ens\textbf{a}, s\=ens\textbf{um eris}
    & s\=ens\textbf{i}, s\=ens\textbf{ae}, s\=ens\textbf{a eritis} \\\padline
  3 & s\=ens\textbf{us}, s\=ens\textbf{a}, s\=ens\textbf{um erit}
    & s\=ens\textbf{i}, s\=ens\textbf{ae}, s\=ens\textbf{a erunt} \par \\\hline
\end{verbchart}
