\section{The active voice}
Latin verbs can be categorized into four \textit{conjugations} which are
quickly discerned by their differing infinitive forms (i.e., look at the
second principle part to determine the theme vowel).  This section will
enumerate the regular endings for each conjugations, and list some more
common deviations.

The conjugations have been split into to overarching sections
corresponding to the two Latin voices, \emph{active} and \emph{passive}.
This is necessitated owing to the consistently differing personal
endings for present-tense forms and by the completely different
construction of passive-voice perfect-tense forms mandating the use of
a helping form of \emph{esse}.

This section will describe forms constructed using the active voice.

\subsection{Present-tense paradigm forms}
The present-tense system, based on the first principal part, is mostly
irregular, with degrees of variance across the conjugations.  The only
endings common to all conjugations are given here, namely the
\textit{present active indicative} and the 
\textit{present passive indicative}, both of which are seen often as
ending suffixes.

Before jumping into the individual conjugations, presented here are the
personal endings common to all conjugations.

\begin{verbchart}{Personal endings}{present}{active}{indicative}
  1 & -\textbf{o}, -\textbf{m}  & -\textbf{mus} \\\hline
  2 & -\textbf{s}               & -\textbf{tis} \\\hline
  3 & -\textbf{t}               & -\textbf{nt}  \\\hline
\end{verbchart}

% Need to describe the imperfect and future forms.  I didn't want
% to but it won't take that long and it's not cool just having
% the forms here without explaining fuck all about them.

\subsection{Perfect-tense paradigm forms}
The perfect-tense system, based on the third principal part, is regular
in that the same rules apply across all verb conjugations.  Thus, all
the perfect-tense endings are given here.

\begin{verbchart}{Personal endings}{perfect}{active}{indicative}
  1 & -\textbf{i}       & -\textbf{imus}    \\\hline
  2 & -\textbf{ist\=i}  & -\textbf{istis}   \\\hline
  3 & -\textbf{it}      & -\textbf{\=erunt} \\\hline
\end{verbchart}

\begin{verbchart}{Personal endings}{perfect}{active}{subjunctive}
  1 & -\textbf{erim}    & -\textbf{er\=imus}  \\\hline
  2 & -\textbf{er\=is}  & -\textbf{er\=itis}  \\\hline
  3 & -\textbf{erit}    & -\textbf{erint}     \\\hline
\end{verbchart}

\begin{verbchart}{Personal endings}{pluperfect}{active}{indicative}
  1 & -\textbf{eram}    & -\textbf{er\=amus}  \\\hline
  2 & -\textbf{er\=as}  & -\textbf{er\=atis}  \\\hline
  3 & -\textbf{erat}    & -\textbf{erant}     \\\hline
\end{verbchart}

\begin{verbchart}{Personal endings}{pluperfect}{active}{subjunctive}
  1 & -\textbf{issem}  & -\textbf{iss\=emus}  \\\hline
  2 & -\textbf{iss\=es}  & -\textbf{iss\=etis}  \\\hline
  3 & -\textbf{isset}  & -\textbf{issent}  \\\hline
\end{verbchart}

The future perfect active indicative looks much like the perfect
active subjunctive, with the exceptions of the first person
singular, and the length of the \textit{i}, which changes
where the accent is placed.
\begin{verbchart}{Personal endings}{future perfect}{active}{indicative}
  1 & -\textbf{er\=o} & -\textbf{erimus}  \\\hline
  2 & -\textbf{eris} & -\textbf{eritis}  \\\hline
  3 & -\textbf{erit} & -\textbf{erint}  \\\hline
\end{verbchart}

\subsection{The First Conjugation (Long-\=a verbs)}
First conjugation verbs are characterized by the theme vowel, \=a.  Unlike
all other conjugations, the subjunctive is formed with a long \=e.

\paragraph{Present-tense paradigm forms}

\begin{verbchart}{d\=o, d\=are, ded\=i, datum}{present}{active}{indicative}
  1 & d\textbf{\=o}    & d\=a\textbf{mus} \\\hline
  2 & d\=a\textbf{s}   & d\=a\textbf{tis} \\\hline
  3 & da\textbf{t}     & da\textbf{nt} \\\hline
\end{verbchart}

\begin{verbchart}{am\=o, am\=are, am\=avi, am\=atum}{present}{active}{subjunctive}
  1 & am\textbf{em}    & am\textbf{\=emus} \\\hline
  2 & am\textbf{\=es}  & am\textbf{\=etis} \\\hline
  3 & am\textbf{\=et}  & am\textbf{ent} \\\hline
\end{verbchart}

\begin{verbchart}{d\=o, d\=are, ded\=i, datum}{future}{active}{indicative}
  1 & d\=a\textbf{b\=o}    & d\=a\textbf{bimus} \\\hline
  2 & d\=a\textbf{bis}     & d\=a\textbf{bitis} \\\hline
  3 & d\=a\textbf{bit}     & d\=a\textbf{bunt} \\\hline
\end{verbchart}

\begin{verbchart}{d\=o, d\=are, ded\=i, datum}{imperfect}{active}{indicative}
  1 & d\=a\textbf{bam}   & d\=a\textbf{b\=amus} \\\hline
  2 & d\=a\textbf{b\=as} & d\=a\textbf{b\=atis} \\\hline
  3 & d\=a\textbf{bat}   & d\=a\textbf{bant} \\\hline
\end{verbchart}

\begin{verbchart}{am\=o, am\=are, am\=avi, am\=atum}{imperfect}{active}{subjunctive}
  1 & am\=ar\textbf{em}    & am\=ar\textbf{\=emus} \\\hline
  2 & am\=ar\textbf{\=es}  & am\=ar\textbf{\=etis} \\\hline
  3 & am\=ar\textbf{\=et}  & am\=ar\textbf{ent} \\\hline
\end{verbchart}

\paragraph{Perfect-tense paradigm forms}

\begin{verbchart}{am\=o, am\=are, am\=av\=i, am\=atum}{perfect}{active}{indicative}
  1 & am\=av\textbf{i}       & am\=av\textbf{imus}    \\\hline
  2 & am\=av\textbf{ist\=i}  & am\=av\textbf{istis}   \\\hline
  3 & am\=av\textbf{it}      & am\=av\textbf{\=erunt} \\\hline
\end{verbchart}

\begin{verbchart}{am\=o, am\=are, am\=av\=i, am\=atum}{perfect}{active}{subjunctive}
  1 & am\=av\textbf{erim}    & am\=av\textbf{er\=imus}  \\\hline
  2 & am\=av\textbf{er\=is}  & am\=av\textbf{er\=itis}  \\\hline
  3 & am\=av\textbf{erit}    & am\=av\textbf{erint}     \\\hline
\end{verbchart}

\begin{verbchart}{am\=o, am\=are, am\=av\=i, am\=atum}{pluperfect}{active}{indicative}
  1 & am\=av\textbf{eram}    & am\=av\textbf{er\=amus}  \\\hline
  2 & am\=av\textbf{er\=as}  & am\=av\textbf{er\=atis}  \\\hline
  3 & am\=av\textbf{erat}    & am\=av\textbf{erant}     \\\hline
\end{verbchart}

\begin{verbchart}{am\=o, am\=are, am\=av\=i, am\=atum}{pluperfect}{active}{subjunctive}
  1 & am\=av\textbf{issem}  & am\=av\textbf{iss\=emus}  \\\hline
  2 & am\=av\textbf{iss\=es}  & am\=av\textbf{iss\=etis}  \\\hline
  3 & am\=av\textbf{isset}  & am\=av\textbf{issent}  \\\hline
\end{verbchart}

\begin{verbchart}{am\=o, am\=are, am\=av\=i, am\=atum}{future perfect}{active}{indicative}
  1 & am\=av\textbf{er\=o} & am\=av\textbf{erimus}  \\\hline
  2 & am\=av\textbf{eris} & am\=av\textbf{eritis}  \\\hline
  3 & am\=av\textbf{erit} & am\=av\textbf{erint}  \\\hline
\end{verbchart}

\subsection{The Second Conjugation (Long-\=e verbs)}
Second conjugations verbs are characterized by the theme vowel, \=e.  Easily
recognizable second conjugation verbs include \textit{vide\=o, vid\=ere,
v\=id\=i, v\=isum}, and \textit{habe\=o, hab\=ere, habu\=i, habitum}.

\paragraph{Present-tense paradigm forms}

\begin{verbchart}{vide\=o, vid\=ere, v\={i}d\=i, v\=isum}{present}{active}{indicative}
  1 & vide\textbf{\=o}   & vid\=e\textbf{mus} \\\hline
  2 & vid\=e\textbf{s}   & vid\=e\textbf{tis} \\\hline
  3 & vide\textbf{t}     & vide\textbf{nt} \\\hline
\end{verbchart}

Unlike the first and third conjugations, and like the fourth conjugation,
the subjunctive for second conjugation verbs is formed with a long \=a 
\textit{after} a shortened theme vowel, in this case short-e.

\begin{verbchart}{habe\=o, hab\=ere, habu\=i, habitum}{present}{active}{subjunctive}
  1 & habe\textbf{am}    & habe\textbf{\=amus} \\\hline
  2 & habe\textbf{\=as}  & habe\textbf{\=atis} \\\hline
  3 & habe\textbf{\=at}  & habe\textbf{ant} \\\hline
\end{verbchart}

\begin{verbchart}{vide\=o, vid\=ere, v\={i}d\=i, v\=isum}{future}{active}{indicative}
  1 & vid\=e\textbf{b\=o}   & vid\=e\textbf{bimus} \\\hline
  2 & vid\=e\textbf{bis}    & vid\=e\textbf{bitis} \\\hline
  3 & vid\=e\textbf{bit}    & vide\textbf{bunt} \\\hline
\end{verbchart}

\begin{verbchart}{vide\=o, vid\=ere, v\={i}d\=i, v\=isum}{imperfect}{active}{indicative}
  1 & vid\=e\textbf{bam}    & vid\=e\textbf{b\=amus} \\\hline
  2 & vid\=e\textbf{b\=as}  & vid\=e\textbf{b\=atis} \\\hline
  3 & vid\=e\textbf{bat}    & vid\=e\textbf{bant} \\\hline
\end{verbchart}

\begin{verbchart}{habe\=o, hab\=ere, habu\=i, habitum}{imperfect}{active}{subjunctive}
  1 & hab\=e\textbf{rem}    & hab\=e\textbf{r\=emus} \\\hline
  2 & hab\=e\textbf{r\=es}  & hab\=e\textbf{r\=etis} \\\hline
  3 & hab\=e\textbf{ret}    & hab\=e\textbf{rent} \\\hline
\end{verbchart}

\paragraph{Perfect-tense paradigm forms}

\begin{verbchart}{vide\=o, vid\=ere, v\=id\=i, v\=isum}{perfect}{active}{indicative}
  1 & v\=id\textbf{i}       & v\=id\textbf{imus}    \\\hline
  2 & v\=id\textbf{ist\=i}  & v\=id\textbf{istis}   \\\hline
  3 & v\=id\textbf{it}      & v\=id\textbf{\=erunt} \\\hline
\end{verbchart}

\begin{verbchart}{vide\=o, vid\=ere, v\=id\=i, v\=isum}{perfect}{active}{subjunctive}
  1 & v\=id\textbf{erim}    & v\=id\textbf{er\=imus}  \\\hline
  2 & v\=id\textbf{er\=is}  & v\=id\textbf{er\=itis}  \\\hline
  3 & v\=id\textbf{erit}    & v\=id\textbf{erint}     \\\hline
\end{verbchart}

\begin{verbchart}{vide\=o, vid\=ere, v\=id\=i, v\=isum}{pluperfect}{active}{indicative}
  1 & v\=id\textbf{eram}    & v\=id\textbf{er\=amus}  \\\hline
  2 & v\=id\textbf{er\=as}  & v\=id\textbf{er\=atis}  \\\hline
  3 & v\=id\textbf{erat}    & v\=id\textbf{erant}     \\\hline
\end{verbchart}

\begin{verbchart}{vide\=o, vid\=ere, v\=id\=i, v\=isum}{pluperfect}{active}{subjunctive}
  1 & v\=id\textbf{issem}  & v\=id\textbf{iss\=emus}  \\\hline
  2 & v\=id\textbf{iss\=es}  & v\=id\textbf{iss\=etis}  \\\hline
  3 & v\=id\textbf{isset}  & v\=id\textbf{issent}  \\\hline
\end{verbchart}

\begin{verbchart}{vide\=o, vid\=ere, v\=id\=i, v\=isum}{future perfect}{active}{indicative}
  1 & v\=id\textbf{er\=o} & v\=id\textbf{erimus}  \\\hline
  2 & v\=id\textbf{eris} & v\=id\textbf{eritis}  \\\hline
  3 & v\=id\textbf{erit} & v\=id\textbf{erint}  \\\hline
\end{verbchart}

\subsection{The Third Conjugation (Short-e verbs)}
The third conjugation of Latin verbs is arguably the least regular, and is
characterized by the theme vowel short-e.  Notably, this short-e tends to
manifest itself as a short-i, as in the present active indicative.

\paragraph{Present-tense paradigm forms}

\begin{verbchart}{d\=ic\=o, d\=icere, d\=ix\=i, dictum}{present}{active}{indicative}
  1 & d\=ic\textbf{\=o}    & d\=ici\textbf{mus} \\\hline
  2 & d\=ici\textbf{s}     & d\=ici\textbf{tis} \\\hline
  3 & d\=ici\textbf{t}     & d\=icu\textbf{nt} \\\hline
\end{verbchart}

\begin{verbchart}{p\=on\=o, p\=onere, posu\=i, positum}{present}{active}{subjunctive}
  1 & p\=on\textbf{am}    & p\=on\textbf{\=amus} \\\hline
  2 & p\=on\textbf{\=as}  & p\=on\textbf{\=atis} \\\hline
  3 & p\=on\textbf{\=at}  & p\=on\textbf{ant} \\\hline
\end{verbchart}

\begin{verbchart}{p\=on\=o, p\=onere, posu\=i, positum}{future}{active}{indicative}
  1 & pon\textbf{am}    & pon\textbf{\=emus} \\\hline
  2 & pon\textbf{\=es}  & pon\=a\textbf{\=etis} \\\hline
  3 & pon\textbf{et}    & pon\textbf{ent} \\\hline
\end{verbchart}

\begin{verbchart}{d\=ic\=o, d\=icere, d\=ix\=i, dictum}{imperfect}{active}{indicative}
  1 & d\=ic\=e\textbf{bam}    & d\=ic\=e\textbf{b\=amus} \\\hline
  2 & d\=ic\=e\textbf{b\=as}  & d\=ic\=e\textbf{b\=atis} \\\hline
  3 & d\=ic\=e\textbf{bat}    & d\=ic\=e\textbf{bant} \\\hline
\end{verbchart}

\begin{verbchart}{p\=on\=o, p\=onere, posu\=i, positum}{imperfect}{active}{subjunctive}
  1 & p\=one\textbf{rem}    & p\=one\textbf{r\=emus} \\\hline
  2 & p\=one\textbf{r\=es}  & p\=one\textbf{r\=etis} \\\hline
  3 & p\=one\textbf{ret}    & p\=one\textbf{rent} \\\hline
\end{verbchart}

\paragraph{Perfect-tense paradigm forms}

\begin{verbchart}{p\=on\=o, p\=onere, posu\=i, positum}{perfect}{active}{indicative}
  1 & posu\textbf{i}       & posu\textbf{imus}    \\\hline
  2 & posu\textbf{ist\=i}  & posu\textbf{istis}   \\\hline
  3 & posu\textbf{it}      & posu\textbf{\=erunt} \\\hline
\end{verbchart}

\begin{verbchart}{p\=on\=o, p\=onere, posu\=i, positum}{perfect}{active}{subjunctive}
  1 & posu\textbf{erim}    & posu\textbf{er\=imus}  \\\hline
  2 & posu\textbf{er\=is}  & posu\textbf{er\=itis}  \\\hline
  3 & posu\textbf{erit}    & posu\textbf{erint}     \\\hline
\end{verbchart}

\begin{verbchart}{p\=on\=o, p\=onere, posu\=i, positum}{pluperfect}{active}{indicative}
  1 & posu\textbf{eram}    & posu\textbf{er\=amus}  \\\hline
  2 & posu\textbf{er\=as}  & posu\textbf{er\=atis}  \\\hline
  3 & posu\textbf{erat}    & posu\textbf{erant}     \\\hline
\end{verbchart}

\begin{verbchart}{p\=on\=o, p\=onere, posu\=i, positum}{pluperfect}{active}{subjunctive}
  1 & posu\textbf{issem}  & posu\textbf{iss\=emus}  \\\hline
  2 & posu\textbf{iss\=es}  & posu\textbf{iss\=etis}  \\\hline
  3 & posu\textbf{isset}  & posu\textbf{issent}  \\\hline
\end{verbchart}

\begin{verbchart}{p\=on\=o, p\=onere, posu\=i, positum}{future perfect}{active}{indicative}
  1 & posu\textbf{er\=o} & posu\textbf{erimus}  \\\hline
  2 & posu\textbf{eris} & posu\textbf{eritis}  \\\hline
  3 & posu\textbf{erit} & posu\textbf{erint}  \\\hline
\end{verbchart}

\subsection{The Third\textit{-io}}
A subset of the third conjugation, the third-io can
be thought of as part third- and part fourth-conjugation.
It retains the theme vowel short-e but places an i in
some forms where it is not expected for a third conjugation
verb.

\paragraph{Present-tense paradigm forms}

\begin{verbchart}{capi\=o, capere, c\=ep\=i, captum}{present}{active}{indicative}
  1 & capi\textbf{\=o}  & capi\textbf{mus} \\\hline
  2 & capi\textbf{s}    & capi\textbf{tis} \\\hline
  3 & capi\textbf{t}    & capiu\textbf{nt} \\\hline
\end{verbchart}

\begin{verbchart}{capi\=o, capere, c\=ep\=i, captum}{present}{active}{subjunctive}
  1 & capi\textbf{am}   & capi\textbf{\=amus} \\\hline
  2 & capi\textbf{\=as} & capi\textbf{\=atis} \\\hline
  3 & capi\textbf{at}   & capi\textbf{ant} \\\hline
\end{verbchart}

\begin{verbchart}{capi\=o, capere, c\=ep\=i, captum}{future}{active}{indicative}
  1 & capi\textbf{am}     & capi\textbf{\=emus} \\\hline
  2 & capi\textbf{\=es}   & capi\textbf{\=etis} \\\hline
  3 & capi\textbf{et}     & capi\textbf{ent} \\\hline
\end{verbchart}

\begin{verbchart}{capi\=o, capere, c\=ep\=i, captum}{imperfect}{active}{indicative}
  1 & capi\=e\textbf{bam}   & capi\=e\textbf{b\=amus} \\\hline
  2 & capi\=e\textbf{b\=as} & capi\=e\textbf{b\=atis} \\\hline
  3 & capi\=e\textbf{bat}   & capi\=e\textbf{bant} \\\hline
\end{verbchart}

\begin{verbchart}{capi\=o, capere, c\=ep\=i, captum}{imperfect}{active}{subjunctive}
  1 & cape\textbf{rem}    & cape\textbf{r\=emus} \\\hline
  2 & cape\textbf{r\=es}  & cape\=a\textbf{r\=etis} \\\hline
  3 & cape\textbf{ret}    & cape\textbf{rent} \\\hline
\end{verbchart}

\paragraph{Perfect-tense paradigm forms}

\begin{verbchart}{capi\=o, capere, c\=ep\=i, captum}{perfect}{active}{indicative}
  1 & c\=ep\textbf{i}       & c\=ep\textbf{imus}    \\\hline
  2 & c\=ep\textbf{ist\=i}  & c\=ep\textbf{istis}   \\\hline
  3 & c\=ep\textbf{it}      & c\=ep\textbf{\=erunt} \\\hline
\end{verbchart}

\begin{verbchart}{capi\=o, capere, c\=ep\=i, captum}{perfect}{active}{subjunctive}
  1 & c\=ep\textbf{erim}    & c\=ep\textbf{er\=imus}  \\\hline
  2 & c\=ep\textbf{er\=is}  & c\=ep\textbf{er\=itis}  \\\hline
  3 & c\=ep\textbf{erit}    & c\=ep\textbf{erint}     \\\hline
\end{verbchart}

\begin{verbchart}{capi\=o, capere, c\=ep\=i, captum}{pluperfect}{active}{indicative}
  1 & c\=ep\textbf{eram}    & c\=ep\textbf{er\=amus}  \\\hline
  2 & c\=ep\textbf{er\=as}  & c\=ep\textbf{er\=atis}  \\\hline
  3 & c\=ep\textbf{erat}    & c\=ep\textbf{erant}     \\\hline
\end{verbchart}

\begin{verbchart}{capi\=o, capere, c\=ep\=i, captum}{pluperfect}{active}{subjunctive}
  1 & c\=ep\textbf{issem}  & c\=ep\textbf{iss\=emus}  \\\hline
  2 & c\=ep\textbf{iss\=es}  & c\=ep\textbf{iss\=etis}  \\\hline
  3 & c\=ep\textbf{isset}  & c\=ep\textbf{issent}  \\\hline
\end{verbchart}

\begin{verbchart}{capi\=o, capere, c\=ep\=i, captum}{future perfect}{active}{indicative}
  1 & c\=ep\textbf{er\=o} & c\=ep\textbf{erimus}  \\\hline
  2 & c\=ep\textbf{eris} & c\=ep\textbf{eritis}  \\\hline
  3 & c\=ep\textbf{erit} & c\=ep\textbf{erint}  \\\hline
\end{verbchart}

\subsection{The Fourth Conjugation (Long-\=i verbs)}
The fourth conjugation is characterized by the theme vowel
long-\=i.

\paragraph{Present-tense paradigm forms}

\begin{verbchart}{audi\=o, aud\=ire, aud\=iv\=i, auditum}{present}{active}{indicative}
  1 & audi\textbf{\=o}   & aud\=i\textbf{mus} \\\hline
  2 & aud\=i\textbf{s}   & aud\=i\textbf{tis} \\\hline
  3 & aud\=i\textbf{t}   & audiu\textbf{nt} \\\hline
\end{verbchart}

\begin{verbchart}{veni\=o, ven\=ire, v\=en\=i, ventum}{present}{active}{subjunctive}
  1 & veni\textbf{am}    & veni\textbf{\=amus} \\\hline
  2 & veni\textbf{\=as}  & veni\textbf{\=atis} \\\hline
  3 & veni\textbf{\=at}  & veni\textbf{ant} \\\hline
\end{verbchart}

\begin{verbchart}{audi\=o, aud\=ire, aud\=iv\=i, auditum}{future}{active}{indicative}
  1 & audi\textbf{am}     & audi\textbf{\=emus} \\\hline
  2 & audi\textbf{\=es} & audi\textbf{\=etis} \\\hline
  3 & audi\textbf{et}   & audi\textbf{ent} \\\hline
\end{verbchart}

\begin{verbchart}{audi\=o, aud\=ire, aud\=iv\=i, auditum}{imperfect}{active}{indicative}
  1 & audi\=e\textbf{bam}   & audi\=e\textbf{b\=amus} \\\hline
  2 & audi\=e\textbf{b\=as} & audi\=e\textbf{b\=atis} \\\hline
  3 & audi\=e\textbf{bat}   & audi\=e\textbf{bant} \\\hline
\end{verbchart}

\begin{verbchart}{veni\=o, ven\=ire, v\=en\=i, ventum}{imperfect}{active}{subjunctive}
  1 & veni\textbf{rem}    & veni\textbf{r\=emus} \\\hline
  2 & veni\textbf{r\=es}  & veni\textbf{r\=etis} \\\hline
  3 & veni\textbf{ret}    & veni\textbf{rent} \\\hline
\end{verbchart}

\paragraph{Perfect-tense paradigm forms}

\begin{verbchart}{senti\=o, sent\=ire, s\=ens\=i, s\=ensum}{perfect}{active}{indicative}
  1 & s\=ens\textbf{i}       & s\=ens\textbf{imus}    \\\hline
  2 & s\=ens\textbf{ist\=i}  & s\=ens\textbf{istis}   \\\hline
  3 & s\=ens\textbf{it}      & s\=ens\textbf{\=erunt} \\\hline
\end{verbchart}

\begin{verbchart}{senti\=o, sent\=ire, s\=ens\=i, s\=ensum}{perfect}{active}{subjunctive}
  1 & s\=ens\textbf{erim}    & s\=ens\textbf{er\=imus}  \\\hline
  2 & s\=ens\textbf{er\=is}  & s\=ens\textbf{er\=itis}  \\\hline
  3 & s\=ens\textbf{erit}    & s\=ens\textbf{erint}     \\\hline
\end{verbchart}

\begin{verbchart}{senti\=o, sent\=ire, s\=ens\=i, s\=ensum}{pluperfect}{active}{indicative}
  1 & s\=ens\textbf{eram}    & s\=ens\textbf{er\=amus}  \\\hline
  2 & s\=ens\textbf{er\=as}  & s\=ens\textbf{er\=atis}  \\\hline
  3 & s\=ens\textbf{erat}    & s\=ens\textbf{erant}     \\\hline
\end{verbchart}

\begin{verbchart}{senti\=o, sent\=ire, s\=ens\=i, s\=ensum}{pluperfect}{active}{subjunctive}
  1 & s\=ens\textbf{issem}  & s\=ens\textbf{iss\=emus}  \\\hline
  2 & s\=ens\textbf{iss\=es}  & s\=ens\textbf{iss\=etis}  \\\hline
  3 & s\=ens\textbf{isset}  & s\=ens\textbf{issent}  \\\hline
\end{verbchart}

\begin{verbchart}{senti\=o, sent\=ire, s\=ens\=i, s\=ensum}{future perfect}{active}{indicative}
  1 & s\=ens\textbf{er\=o} & s\=ens\textbf{erimus}  \\\hline
  2 & s\=ens\textbf{eris} & s\=ens\textbf{eritis}  \\\hline
  3 & s\=ens\textbf{erit} & s\=ens\textbf{erint}  \\\hline
\end{verbchart}
