\section{Deponent verbs}
A subset of Latin verbs of which a student of the language
ought to be wary are called \emph{deponent}, which ``put
aside'' active endings in preference for passive ones.  In
effect, these are verbs which look passive but are in
actuality active, and must be translated accordingly.

Deponents are dinstinguished by their principal parts,
which will appear conspicuously passive.  Identifying a
verb listing whose principal parts are all passive (and
that there are only three such principal parts instead of
the usual four, but this can also be an indicator of an
irregular) is tantamount to finding a deponent.

Particularly sharp readers will have noted that a
consequence of active meanings parading behind passive
forms means that deponent verbs will never be translated
as passive, even if they always look passive.  Put
another way, the passive voice can only be generated
for verbs with active forms.

% Do I really want to do a full conjugation for these?
% Would it make sense to show examples for each conjugation
% comprising active and passive conjugation of a model
% verb plus an active (passive-looking) conjugation for
% a model deponent?
%\subsection{First conjugation}
%
%\subsection{Second conjugation}
%
%\subsection{Third conjugation}
%
%\subsection{The third-\emph{io}}
%
%\subsection{Fourth conjugation}
%

\subsection{The imperfect subjunctive}
Generally, to form the imperfect subjunctive, one takes the
present active infinitive, then adds the corresponding
personal endings.  When dealing with deponents however, the
present active infinitive is passive in form.  Thus, some
trickery is required in order to ``restore'' the active
form, after which passive personal endings can be applied
to said.

\subsubsection{First, second, and fourth conjugations}
The process for ``restoring'' the active form is not tough
for first, second, and fourth conjugation verbs.  Take the
second principal part, drop the long \=i and replace it
with a short e.

Do note that this form will not exist by itself, and only
makes sense to form when forming the imperfect subjunctive.

% Examples:
%
% hort\=ar\=i -> hort\=ar_ -> hort\=are
% conjugate (add to table):
% hort\=arer, hort\=ar\=eris, hort\=ar\=eture,
% hort\=ar\=emur, hort\=ar\=emin\=i, hort\=ar\=entur
%
% ver\=er\=i -> ver\=ere
% conjugate table
%
% part\=ir\=i -> part\=ire
% conjugate table

\subsubsection{Third, and third-io}
The third and third-io require somewhat more involved
trickery.  Since the passive infinitive for the third and
third-io conjugations drop the -ere ending present in the
active forms and replace this with a long \=i for the
passive form, we must restore this ending to craft our
intermediary form before forming the imperfect subjunctive.

Again, it's worth noting that this form is best thought of
as ``imaginary'' as it will not exist outside the formation
of the imperfect subjunctive.

% Examples:
%
% sequ\=i -> sequere
% conjugate (add to table):
% sequerem, sequer\=eris, sequer\=etur, 
% sequer\=emur, sequer\=emin\=i, sequerentur
% compare with present active infinitive? not if done above

\subsection{Semi-deponents}
As if things weren't confusing enough, the Latin student
must also be wary of so-called \emph{semi-deponent} verbs,
whose active translations appear active in some tenses and
passive in others.  Again, identifying that some principal
parts are active while others passive is indication that a
verb is semi-deponent.

% Examples:
%
% dare:
% aude\=o, aud\=ere,  ausus sum
% ^^^^^^active^^^^^  ^^passive^^
%
% rejoice:
% gaude\=o, gaud\=ere, gav\=isus sum
% ^^^^^^^active^^^^^^^  ^^passive^^
%
% be accustomed:
% sole\=o, sol\=ere, solitus sum
% ^^^^^^active^^^^^  ^^passive^^
%
% trust in:
% c\=onf\=id\=o, c\=onf\=idere, c\=onf\=isus sum
% ^^^^^^^^^^^active^^^^^^^^^^^  ^^^^passive^^^^^
% 

Both deponents and semi-deponents are not called out as
such in this reference, and it is the reader's
responsibility to identify these verbs accordingly, as
per their principal parts.

\subsubsection{Participles}
Deponents, with one exception, only have active 
participles.  The past participle, the future active
participle, and the present active participle all look
passive, but are translated as active in meaning, in line
with our dealings with deponent verbs thus far.

% Example:
%
% speak, converse
% loquor, loqu\=i, loc\=utus sum
%
% having spoken
% loc\=utus, -a, -um
%
% going to speak
% loc\=ut\=urus, -a, -um
%
% speaking
% loqu\=ens

Finally, we now come across a form for a deponent verb that
actually has a passive meaning to go with its passive form.
The future passive participle, formed by adding the ending
\emph{-endus} to the present stem.

% Example:
%
% must be spoken
% loquendus, -a, -um

\subsubsection{Infinitives}
It is worth calling out that deponent infinitive forms,
while regular, follow a perhaps unexpected pattern.  The
present and perfect tense infinitives appear passive as
expected.  The future active infinitive, however, is
active in form and meaning.

% examples for every conjugation?  Otherwise what's the
% point of this reference anyway?

% both passive:
% loqu\=i -> to speak
% loc\=utus esse -> to have spoken

% active:
% loc\=ut\=urus esse -> to be going to speak
